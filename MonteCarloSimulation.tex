\documentclass[a4paper,11pt]{article}

\usepackage{a4wide}
\usepackage{amsmath}
\usepackage{hyperref}

\begin{document}
\title{Monte Carlo simulations of the Ising model}
\author{Jakob Schiøtz}
\date{\today}
\maketitle

The computer exercise for this week uses Monte Carlo simulations to
study the Ising model, with a particular focus on fluctuations.

\section{The computer code}

The computer code is in the assocated file \texttt{IsingModel.ipynb},
it is a Jupyter notebook.  If you are not set up to run Jupyter
notebooks on your own computer, you can run it in the cloud,
instructions for both are in the Appendix.

\section{Units}

The simulations assume units where $J=1$ and $k=1$.  We thus measure
both energies and temperatures in units of $j$.

\section{Part I: Manual investigation}

The first part of the Notebook investigates what happens at a single
temperature.  Look at the function \texttt{initialize()}, it sets up a
spin lattice of a given size.  The function \texttt{simulate()} runs
the actual Ising Model simulation.

\begin{itemize}
\item Look at the code of the two functions, and convince yourself
  that it makes sense.
\item Run MC simulations at $T=1$, $T=2$ and $T=3$.
  \begin{itemize}
  \item Look at the output states, what are the qualitative
    differences.
  \item Look at the graphs of energy and magnetization versus step.
    Is the simulation long enough?  In the next part, we are going to
    make averages over the second half of the data produced, so a
    transient should be limited to the first half.
  \end{itemize}
\end{itemize}

\section{Part II: Temperature variations}

The second part of the code contains a loop that loops over
temperatures.  The temperatures are not equidistance, they are closer
near the expected transition temperature, in order to not waste too
much time on aquiring data at less interesting temperatures.

\begin{itemize}
\item Run the temperature loop, and plot energy and magnetization
  versus temperature.  Estimate the transition temperature.
\item The system is more easily caught in a local minimum when
  starting with a random spin configuration than when starting with
  all spin up.  For that reason, this part of the code does the
  latter.
  \begin{itemize}
  \item Do you think this might bias the result when you try to
    estimate the transition temperature?
  \item Would starting with a random configuration also bias the
    result?
    \item What could be done to limit the effect of this bias?  At
      what cost?
  \end{itemize}
\item The heat capacity $C_v$ can be estimated from the fluctuations
  in the energy (note that you need the fluctuations of the total
  energy, not the energy per site.  The variable \texttt{nsites}
  contains the number of sites).  Calculate and plot the heat capacity
  versus temperature.
\item In a similar way, calculate and plot the magnetic susceptibility
  $\chi$ from the fluctuations in the magnetization.  It only makes
  sense above the transition temperature (why?)
\item Why do both $C_v$ and $\chi$ increase dramatically near the
  transition temperature?
\end{itemize}

\section{Finite-size effects}

Use the first section of the notebook to look at the state produced by
simulations close to the critical temperature.  You will see that
large clusters of the same spin form as fluctuations, they become
smaller as you move away from the critical tempeture.

\begin{itemize}
\item How do you expect the finite size of the system to affect the
  simulations?
 \item What happens with the critical temperature if you run with a
   much smaller system, e.g. $4\times4$ spins instead of $20\times20$?
   Reducing the system size is easier than increasing it, due to how
   the computational time scales.
\end{itemize}


\section{Finite field - critical exponents}

The simulation function takes an extra parameter \texttt{H} specifying
an external field (per default it is off).  Set the temperature to the
(theoretical) critical temperature $T=2.269$ and calculate the average
magnetisation for magnetic fields of $H = 0.01, 0.02, 0.04, 0.008$ and
$0.016$.  Plot the data so you can use it to estimate the critical
exponent in equation (5.88): $\left|m\right| \sim \left|H\right|^{1/\delta}$


\clearpage
\appendix

\section*{Appendix}

\section{Running the code in Google Colab}
\label{sec:running}

To run the code in your browser in Google Colab, you need a Google
account.  Log in to Google Colab at
\url{https://colab.research.google.com/} and then click on the Open In
Colab badge in DTU Learn.  This will copy the notebook from my GitHub
account to your Colab workspace.

\section{Running on your own computer}

This requires a working Python installation on your laptop /
computer.  On Windows Anaconda is popular (and good).  On Linux,
Python comes with the OS but you may need to install notebook
packages.  On MacOS install using Homebrew or download directly from
Python.org, using Anaconda on a mac is not recommended.

\subsection{Running in Visual Studio}

If you are using Visual Studio, you can download the notebook file
from DTU Learn and open it in Visual Studio.  VS may suggest that you
install extensions for notebooks, you should accept that.  You run the
cells by pressing Shift-Enter in the cell.

\subsection{Running in your browser}

If you use Anaconda on Windows, open an Anaconda terminal from your
start menu.  If you use something else, open a terminal / command
window.  In that window, run the command \texttt{jupyter notebook}.  A
notebook server should start and a window open in your browser.
Download the notebook file from DTU Learn, and open it.  You run the
cells by pressing Shift-Enter in the cell.

\end{document}
%%% Local Variables:
%%% mode: latex
%%% TeX-master: t
%%% End:
