
%%% NOTE:
%%%
%%% The figures for this document are created in the notebook
%%% SimulationGuide.ipynb
%%%


\documentclass[a4paper,11pt]{article}

\usepackage{a4wide}
\usepackage{amsmath}
\usepackage[hidelinks]{hyperref}
\usepackage{siunitx}
\usepackage{graphicx}

\usepackage[sort&compress,numbers]{natbib}
\bibliographystyle{apsrev4-1}
\usepackage{doi}

\begin{document}
\title{Molecular Dynamics simulations of the Lennard-Jones gas}
\author{Jakob Schiøtz}
\date{\today}
\maketitle

\begin{abstract}
  The computer exercise for this week uses Molecular Dynamics
  simulations to study the Lennard Jones gas, and how it deviates from
  the ideal gas.
\end{abstract}

\tableofcontents
\clearpage

\section{Theory}

\subsection{Molecular Dynamics (MD)}

In molecular dynamics (MD), you numerically solve Newton's second law
for the atoms in a system (we thus perform classical dynamics,
ignoring that atoms are really quantum objects).  For this purpose it
is most practical to write Newton's second law as a set of coupled
differential equations in the positions $\vec r$ and velocities $\vec
v$:
\begin{align}
  \label{eq:1}
  {d\vec v \over dt} &= {1 \over m} \vec F(\vec r) \\
  {d\vec r \over dt} &= \vec v 
\end{align}

In the computer, we discretize this by introducing a discrete time
step, $\Delta t$.  Doing this naively leads to the Euler algorithm, which
\emph{does not work} in practice:
\begin{align}
  \label{eq:Euler}
  \vec v(t + \Delta t) &= \vec v(t) + {1 \over m} \vec F(\vec r(t))
                         \Delta t\\
  \vec r(r + \Delta t) &= \vec r(t) + \vec v(t) \Delta t
\end{align}

Attempting to use this algorithm for molecular dynamics will result in
the total energy of the system growing without bounds, even for very
small time-steps.  The problem is that the numerical error is always
in the same direction.

This is illustrated with a very simple example, a harmonic oscillator
(a mass on a spring).  The exact solution of the motion of the
harmonic oscillator is that position and momentum both change as
$\sin(t)$, but $\SI{90}{\degree}$ out of phase, i.e. a plot of
$\bigl( r(t), p(t) \bigr)$ will be an ellipse.  But Euler's algorithm
approximates this by a function that is stepwise linear, and every
step is tangent to a (scaled) ellipse, see Figure
\ref{fig:algorithms}(A).  This leads to a systematic increase in the
total energy of the system that can be reduced by reducing the time
step, but not eliminated.
\begin{figure}
  \centering
  \includegraphics[width=\linewidth]{MD_Algorithms.png}
  \caption{Comparison of MD algorithms for a 1D harmonic oscillator.
    Both position (r) and momentum (p) vary sinusoidally, and the
    system traces a circle in phase space (blue circles). (A)
    illustrates the Euler algorithm.  Each time step is tangent to the
    true motion, and therefore moves the system to larger energies
    (illustrated with the grey circles).  (B) illustrates the Verlet
    algorithm, the time step uses halfway values and becomes a chord
    on the circle.  For the harmonic oscillator, the step always hits
    the circle exactly.}
  \label{fig:algorithms}
\end{figure}

\subsubsection{The Velocity Verlet algorithm}
\label{sec:veloc-verl-algor}

The fundamental reason why the Euler algoritm fails, is that it is not
symmetric in time as the equations of motion are.  The Euler algorithm
uses the quantities at the start of the time step to update position
and momentum, a time reversal of this algoritm would use the
quantities at the end of the time step, and thus not be the same
algorithm.

A simple fix is to use the velocity at the middle of the time step to
update the positions, and the average of the force at the beginning
and the end of the time step to update the velocity.  This leads to
the \emph{Velocity Verlet algorithm}:\footnote{The algorithm was first
  used by J. P. Delambre in 1791 \cite{WikiVerlet}, but was formally described in the
  context of molecular dynamics by Loup Verlet in 1967 \cite{Verlet1967}.
  Verlet's formulation did not explicitly use the velocity but instead
  used the position at the two previous time steps.  Using the
  velocity explicitly makes the algorithm clearer and is referred to
  as the \emph{Velocity Verlet} algorithm \cite{dfrenkel2002:mc}.}
\begin{align}
  \label{eq:Verlet}
  \vec v(t + \frac12 \Delta t) &= \vec v(t) + {1 \over m} \vec F(\vec r(t))
                         {\Delta t \over 2}\\
  \vec r(r + \Delta t) &= \vec r(t) + \vec v(t + \frac12 \Delta t)
                         \Delta t\\
  \vec v(t + \Delta t) &= \vec v(t + \frac12 \Delta t) + {1 \over m}
                         \vec F(\vec r(t + \Delta t))
                         {\Delta t \over 2}
\end{align}

The Velocity Verlet algorithm shares many of the symmetries of
Hamiltonian mechanics, and therefore displays great stability in
molecular dynamics applications.  For example, the time reversibility
prevents a systematic drift towards higher or lower energy.  Much more
accurate methods are available for integrating coupled differential
equations, including the well-known Runge-Kutta methods, but the
latter are not time-reversible and although they display better
short-term energy conservation than the Verlet algorithm, over long
simulation times algorithms like Runge-Kutta will display a systematic
breaking of energy conservation.


\subsection{The Lennard-Jones potential and the Lennard-Jones fluid}

Molecular dynamics require the force on all atoms, and that requires a
model of the atomic interactions.  Such models can range from simple
harmonic forces to full quantum mechanical calculations.  In this
exercises we will use the simplest possible model that resenbles
physical interatomic forces (at least qualitatively): the
Lennard-Jones (LJ) potential.

The LJ potential assumes that all interactions between atoms are
pair-wise forces.\footnote{This is a decent approximation for gasses,
  but is pretty bad for solids, in particular metals.}
The force consists of two parts, an attractive part that scales as
$r^{-6}$ just like van der Waals forces between atoms, and a repulsive
part that scales as $r^{-12}$ and represents the Pauli repulsion once
the atoms get close enough that their electrons overlap.  Here $r$
should be understood as the distance between atoms. The full
form is thus $U(r) = A r^{-12} - B r^{-6}$, but is normally rewritten
as
\begin{equation}
  \label{eq:3}
  U(r_{ij}) = 4 \varepsilon \left( \left({\sigma \over r_{ij}}
    \right)^{-12} - \left({\sigma \over r_{ij}}
    \right)^{-6} \right)
\end{equation}
where $\varepsilon$ is the bond energy and $\sigma$ is a
characteristic size of the atom.  The energy is minimal (and takes the
value $-\varepsilon$) when the distance is $r_0 = 2^{1/6} \sigma
\approx 1.12 \,\sigma$.  The total energy thus becomes
\begin{equation}
  \label{eq:2}
  U = \frac12 \sum_{i,j} U(r_{ij})
\end{equation}
where $i$ and $j$ runs over all atoms, the factor $\frac12$
compensates for counting all pairs twice in the sum, and $r_{ij}$ is
the scalar distance between atoms $i$ and $j$.  The force on an atom
$k$ is of course given by $\vec F_k = - \nabla_k U$, where the
derivative is taken with respect to the position of atom $k$.

The LJ potential is a reasonable description of the interactions
between ideal gas atoms, and has been carefully studied both
theoretically and numerically.  Figure \ref{fig:phasediagram} shows a
phase diagram for the LJ system, displaying both a solid, a liquid and
a vapor phase.
\begin{figure}
  \centering
  \includegraphics[width=0.7\linewidth]{LJ_PhaseDiagram.png}
  \caption{Phase diagram for Lennard-Jones systems, as taken from
    Wikipedia \cite{WikiPhase}.  The phase diagram shows phases and
    phase coexistence for different temperatures and densities, and is
    valid for a three-dimensional LJ system.  A two-dimensional LJ
    system will have a similar phase diagram, but the transition
    temperatures will be around a factor of two lower, as each atom
    has fewer neighbors in the condensed phases.}
  \label{fig:phasediagram}
\end{figure}

\subsection{Pressure in 2D simulations}
\label{sec:pressure}

The simulations in this execises are made in two dimensions, to reduce
computational cost and to facilitate visualization.  In two
dimensions, the equations of state become
\begin{align}
  \label{eq:ideal}
  p A &= N k T\\
  E_{kin} &= N k T 
\end{align}
where the area $A$ replaces the volume.  Pressure thus has units of
force per length (instead of per area).  In the kinetic energy, the
factor $\frac32$ becomes $\frac22$ as there are only two degrees of
freedom per atom instead of three.

Pressure in molecular dynamics can be calculated in different ways.
The most popular is called the ``virial method,'' where it is assumed that
there is an ideal-gas contribution and a contribution from the
interatomic forces.  In this computer exercise we will take a more
direct approach.  We enclose the system in a box, whenever an atom
attempts to move outside the box we reflect it, and record the
momentum transferred by the box.  The force on the box from such a
collision is thus $F \delta t = \Delta p$, where $\delta t$ is some
collision time.  The time averaged force on the box thus becomes
\begin{equation}
  \label{eq:5}
  \left< F \right> = {1 \over t} \sum F \delta t = {1 \over t} \sum
  \Delta p
\end{equation}
where the sums are sums over all collision events, and $t$ is the
elapsed time.  The pressure is obtained by dividing by the surface length of
the computational box (a 2D box has a surface length, not a surface
area).


\section{The computer code}

The computer code is in the assocated file \texttt{MolecularDynamics.ipynb},
it is a Jupyter notebook.  Unlike the previous computer exercise, it
will not work well in the cloud, as the code needs to be able to open
a window on your local computer to display animated graphs of the
simulation.

The Notebook contains two parts.  The first defines the code used, we
have chosen not to use preexisting MD packages, so you have the option
of inspecting the code if you feel inclined to understand how it
works.  The main objects defined are

\begin{description}
\item[Atoms] An object containing the data of the (2d) atoms.  It
  contains methods to intialize the positions (randomly) and to set
  initial velocities according to a Maxwell-Boltzmann distribution.
  This class also has methods to calculate the energy of the system
  and the forces on all atoms.
\item[ReflectingBoundaries] Handles the reflecting boundaries and
  keeps track of the momentum transferred to the boundaries, so the
  pressure can be calculated.
\item[VelocitVerlet] The Velocity Verlet algorithm.
\end{description}

The second part is running the actual simulation in a loop.  We
perform 200\,000 time steps in total, the outer loop runs 1000 times
and for each iteration 200 time steps are performed, then data about
the system is collected, and the plot is updated. 

\section{Units}

Since real atoms are not confined to 2D, it does not make much sense
to pick Lennard-Jones parameters for a real gas.  Nevertheless, the
parameters have been set to somewhat sensible values if we assume that
energies (and $kT$) are measures in electron volt, and distances are
in Ångström.  All temperatures are given as $kT$, i.e. in energy units
(or pretend that $k=1$).

The time step in the Velocity Verlet algorithm is set to 0.004, that
will work well for temperatures up to $kT = 0.2$ or slightly above.

\section{Part I: Manual investigation}

Try running the code as it is.  The initial parameters are $\sigma =
\SI{1.0}{\angstrom}$, $\varepsilon = \SI{0.1}{\electronvolt}$ and $kT
= \SI{0.01}{\electronvolt}$.

\begin{itemize}
\item Observe what happens to the kinetic energy as the simulation
  progresses.
\item What happens to the atoms, qualitatively?  Can this explain the
  temperature increase you observe (think in the language of
  thermodynamics).
\item Where are we in the phase diagram of figure \ref{fig:phasediagram}?
\end{itemize}

\section{Part II: Ideal and non-ideal gas}
\label{sec:part-ii}

Increase the temperature so we are well above the temperature where a
liquid forms (the supercritical region) without going so high that the
Verlet dynamics becomes unstable.  The system will equilibrate much
faster, and you can reduce the number of iterations in the outer loop
to 500, to save time.

\begin{itemize}
\item Verify that with these parameters, the pressure is very close to
  what the ideal gas law predicts.
\item The atoms have a finite size (they are disks of radius $\approx
  \sigma$).  For high densities, this will cause deviations from the
  ideal gas law.  Try to estimate how big an effect you would expect.
\item Try rerunning your simulations with significantly smaller system
  sizes (10 or 15 Å), and verify your expectations from above.  Is the
  pressure higher or lower than the ideal gas pressure?  Is the order
  of magnitude of the difference consistent with your expectations?
\item At temperatures closer to the critical point of the liquid-gas
  transitions, the attraction between atoms also begins to influence
  the pressure.  Do you expect this to increase or decrease the
  pressure compared to the ideal gas pressure.  Try so see if you can
  observe this in your simulations.  Be aware that the critical
  temperature is probably approximately half as big for a 2D system as
  for a 3D system.
\end{itemize}

\section{Part III: Radial Distribution Function}

If you have time, you can try to calculate radial distribution
functions (RDFs).  At the end of the notebook there is a cell defining
a \texttt{RadialDistributionFunction} class, you can execute that cell
to define the class, and then insert the lines mentioned in the text
block into the main loop.  You probably need to increase the number of
iterations of the main loop back to 1000, to get good statistics.

\begin{itemize}
\item Plot the RDF for a high temperature and a low temperature
  simulation, for example $kT = 2 \varepsilon$ (that is 0.2) and $kT =
  0.1 \varepsilon$ (that is 0.01).
\item Why is there a distance below which the RDF is zero?
\item Why is there a peak just after that region?
\item Why is that peak much larger at low temperature?
\item \ldots and does this relate to how the pressure is compared to
  the ideal gas pressure?
\end{itemize}

%\nocite{GitHub}

\bibliography{references.bib}

\section*{Availability of code}

The source for this manuscript, the notebook for generating Fig.~1, as
well as the notebooks used in the computer exercises are available at
GitHub:\\
\url{https://github.com/schiotz/DTU_10122}

\end{document}
%%% Local Variables:
%%% mode: latex
%%% TeX-master: t
%%% End:
